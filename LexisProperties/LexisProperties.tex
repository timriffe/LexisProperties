%% BioMed_Central_Tex_Template_v1.06
%%                                      %
%  bmc_article.tex            ver: 1.06 %
%                                       %

%%IMPORTANT: do not delete the first line of this template
%%It must be present to enable the BMC Submission system to
%%recognise this template!!

%%%%%%%%%%%%%%%%%%%%%%%%%%%%%%%%%%%%%%%%%
%%                                     %%
%%  LaTeX template for BioMed Central  %%
%%     journal article submissions     %%
%%                                     %%
%%          <8 June 2012>              %%
%%                                     %%
%%                                     %%
%%%%%%%%%%%%%%%%%%%%%%%%%%%%%%%%%%%%%%%%%

%%%%%%%%%%%%%%%%%%%%%%%%%%%%%%%%%%%%%%%%%%%%%%%%%%%%%%%%%%%%%%%%%%%%%
%%                                                                 %%
%% For instructions on how to fill out this Tex template           %%
%% document please refer to Readme.html and the instructions for   %%
%% authors page on the biomed central website                      %%
%% http://www.biomedcentral.com/info/authors/                      %%
%%                                                                 %%
%% Please do not use \input{...} to include other tex files.       %%
%% Submit your LaTeX manuscript as one .tex document.              %%
%%                                                                 %%
%% All additional figures and files should be attached             %%
%% separately and not embedded in the \TeX\ document itself.       %%
%%                                                                 %%
%% BioMed Central currently use the MikTex distribution of         %%
%% TeX for Windows) of TeX and LaTeX.  This is available from      %%
%% http://www.miktex.org                                           %%
%%                                                                 %%
%%%%%%%%%%%%%%%%%%%%%%%%%%%%%%%%%%%%%%%%%%%%%%%%%%%%%%%%%%%%%%%%%%%%%

%%% additional documentclass options:
%  [doublespacing]
%  [linenumbers]   - put the line numbers on margins

%%% loading packages, author definitions

%\documentclass[twocolumn]{bmcart}% uncomment this for twocolumn layout and comment line below
\documentclass{bmcart}

%%% Load packages
%\usepackage{amsthm,amsmath}
% \RequirePackage{natbib}
\RequirePackage[authoryear]{natbib}% uncomment this for author-year
% bibliography
%\RequirePackage{hyperref}
%\usepackage[utf8]{inputenc} %unicode support
%\usepackage[applemac]{inputenc} %applemac support if unicode package fails
%\usepackage[latin1]{inputenc} %UNIX support if unicode package fails

%%%%%%%%%%%%%%%%%%%%%%%%%%%%%%%%%%%%%%%%%%%%%%%%%%%%%%%%%%%%%%%%%%%%%%%%
% preamble from authors
\usepackage{amsmath}
\usepackage{amsthm}
%\usepackage{appendix}
%\usepackage{amssymb} % for approx greater than
%\usepackage{caption}
\usepackage{placeins} % for \FloatBarrier
\usepackage{graphicx}
\usepackage{subcaption}
\usepackage{longtable}
\usepackage{setspace}
\usepackage{booktabs}
\usepackage{tabularx}
%\usepackage{xcolor,colortbl}
\usepackage{chngpage}
%\usepackage{natbib}
%\bibpunct{(}{)}{,}{a}{}{;} 
\usepackage{url}
\usepackage{nth}
%\usepackage{authblk}
%\usepackage[most]{tcolorbox}
\usepackage[normalem]{ulem}
\usepackage{amsfonts}
% columns for longtable. These work if floats not being
% pushed to end!! commented out to remove compile error
\newcolumntype{C}[1]{>{\centering\let\newline\\\arraybackslash\hspace{0pt}}m{#1}}
\newcolumntype{L}[1]{>{\raggedright\let\newline\\\arraybackslash\hspace{0pt}}m{#1}}
\usepackage{arydshln} % Dashed lines in matrices
\usepackage{lastpage}
%\usepackage[margin=1in]{geometry}
%\doublespacing % for review

%%%%%%%%%%%%%%%%%%%%%%%%%%%%%%%%%%%%%%%%%%%%%%%%%%%%%%%%%%%%%%%%%%%%%%%%%%%%%
% places figures and tables at end. Required for review guidelines!!!
%\usepackage[nolists, nomarkers]{endfloat}
%\DeclareDelayedFloatFlavor{longtable}{table}
%\renewcommand{\efloatseparator}{\mbox{}} % This removes pagebreaks
%%%%%%%%%%%%%%%%%%%%%%%%%%%%%%%%%%%%%%%%%%%%%%%%%%%%%%%%%%%%%%%%%%%%%%%%%%%%%%
% for section 4 math environments
\theoremstyle{definition}
\newtheorem{definition}{Definition}[section]
\newtheorem{theorem}{Theorem}[section]
\newtheorem{proposition}{Proposition}[section]
\newtheorem{corollary}{Corollary}[proposition]
\newtheorem{remark}{Remark}[section]

%%%%%%%%%%%%%%%%%%%%%%%%%%%%%%%%%%%%%%%%%%%%%%%%%%%%%%%%%%%%%%%%%%%%%%%%%%%%%%

\newcommand\ackn[1]{%
  \begingroup
  \renewcommand\thefootnote{}\footnote{#1}%
  \addtocounter{footnote}{-1}%
  \endgroup
}

%%%%%%%%%%%%%%%%%%%%%%%%%%%%%%%%%%%%%%%%%%%%%%%%%%%%%%%
% functions to read in table elements for timelines and graphs table in sec 4 (table 3)
\newcommand{\ttt}[2]{\includegraphics[scale=.45]{{Figures/Tab3#1#2}.pdf}}

% Affiliations in small font size
%\renewcommand\Affilfont{\small}

%\defcitealias{HMD}{HMD 2014}

% junk for longtable caption
%\AtBeginEnvironment{longtable}{\linespread{1}\selectfont}
%\setlength{\LTcapwidth}{\linewidth}

% sort van Raalte properly
% #1: sorting key, #2: prefix for citation, #3: prefix for bibliography
\DeclareRobustCommand{\VAN}[3]{#2} % set up for citation

%%%%%%%%%%%%%%%%%%%%%%%%%%%%%%%%%%%%%%%%%%%%%%%%%
%%                                             %%
%%  If you wish to display your graphics for   %%
%%  your own use using includegraphic or       %%
%%  includegraphics, then comment out the      %%
%%  following two lines of code.               %%
%%  NB: These line *must* be included when     %%
%%  submitting to BMC.                         %%
%%  All figure files must be submitted as      %%
%%  separate graphics through the BMC          %%
%%  submission process, not included in the    %%
%%  submitted article.                         %%
%%                                             %%
%%%%%%%%%%%%%%%%%%%%%%%%%%%%%%%%%%%%%%%%%%%%%%%%%


%\def\includegraphic{}
%\def\includegraphics{}



%%% Put your definitions there:
\startlocaldefs
\endlocaldefs


%%% Begin ...
\begin{document}

%%% Start of article front matter
\begin{frontmatter}

\begin{fmbox}
\dochead{Research}

%%%%%%%%%%%%%%%%%%%%%%%%%%%%%%%%%%%%%%%%%%%%%%
%%                                          %%
%% Enter the title of your article here     %%
%%                                          %%
%%%%%%%%%%%%%%%%%%%%%%%%%%%%%%%%%%%%%%%%%%%%%%

\title{On some properties of generalized Lexis diagrams}

%%%%%%%%%%%%%%%%%%%%%%%%%%%%%%%%%%%%%%%%%%%%%%
%%                                          %%
%% Enter the authors here                   %%
%%                                          %%
%% Specify information, if available,       %%
%% in the form:                             %%
%%   <key>={<id1>,<id2>}                    %%
%%   <key>=                                 %%
%% Comment or delete the keys which are     %%
%% not used. Repeat \author command as much %%
%% as required.                             %%
%%                                          %%
%%%%%%%%%%%%%%%%%%%%%%%%%%%%%%%%%%%%%%%%%%%%%%

\author[
   %addressref={aff1},          % id's of addresses, e.g. {aff1,aff2}
   %corref={aff1},              % id of corresponding address, if any
   %noteref={n1},               % id's of article notes, if any
   email={riffe@mpidr.mpg.de}   % email address
]{\fnm{Tim} \snm{Riffe}}
\author[
   %addressref={aff1},          % id's of addresses, e.g. {aff1,aff2}
   %corref={aff1},              % id of corresponding address, if any
   %noteref={n1},               % id's of article notes, if any
   email={riffe@mpidr.mpg.de}   % email address
]{\fnm{Jonas} \snm{Sch\"oley}}



%\author[
%%   addressref={aff1,aff2},
%   email={john.RS.Smith@cambridge.co.uk}
%]{\inits{JRS}\fnm{John RS} \snm{Smith}}


%%%%%%%%%%%%%%%%%%%%%%%%%%%%%%%%%%%%%%%%%%%%%%
%%                                          %%
%% Enter short notes here                   %%
%%                                          %%
%% Short notes will be after addresses      %%
%% on first page.                           %%
%%                                          %%
%%%%%%%%%%%%%%%%%%%%%%%%%%%%%%%%%%%%%%%%%%%%%%

\begin{artnotes}
%\note{Sample of title note}     % note to the article
\note[id=n1]{Equal contributor} % note, connected to author
\end{artnotes}

\end{fmbox}% comment this for two column layout

%%%%%%%%%%%%%%%%%%%%%%%%%%%%%%%%%%%%%%%%%%%%%%
%%                                          %%
%% The Abstract begins here                 %%
%%                                          %%
%% Please refer to the Instructions for     %%
%% authors on http://www.biomedcentral.com  %%
%% and include the section headings         %%
%% accordingly for your article type.       %%
%%                                          %%
%%%%%%%%%%%%%%%%%%%%%%%%%%%%%%%%%%%%%%%%%%%%%%

\begin{abstractbox}

\begin{abstract} % abstract
%\parttitle{First part title} %if any
Time identities are relationships implied by events and the durations between
them, such at the well-known age-period-cohort relationship. We describe the
construction and structure of such identities.
%\parttitle{Second part title} %if any
\end{abstract}

%%%%%%%%%%%%%%%%%%%%%%%%%%%%%%%%%%%%%%%%%%%%%%
%%                                          %%
%% The keywords begin here                  %%
%%                                          %%
%% Put each keyword in separate \kwd{}.     %%
%%                                          %%
%%%%%%%%%%%%%%%%%%%%%%%%%%%%%%%%%%%%%%%%%%%%%%

\begin{keyword}
\kwd{age structure}
\kwd{formal demography}
\kwd{data visualization}
\kwd{age period cohort}
\end{keyword}

% MSC classifications codes, if any
%\begin{keyword}[class=AMS]
%\kwd[Primary ]{}
%\kwd{}
%\kwd[; secondary ]{}
%\end{keyword}

\end{abstractbox}
%
%\end{fmbox}% uncomment this for twcolumn layout

\end{frontmatter}

\section{Introduction}
Population processes are temporally structured. Sometimes we structure population stocks or flows by more than one element of time at
once. Elements of time may include the moment of observation, the time points of
events such as birth or death, or the time passed between events. Time measures can be categorized into two basic types:
events and durations. Events include any measure at a point \emph{point}
in time. This notion includes period itself, which one may imagine for our
purposes as an infinite set of points. This is not to be confused with the
common notion of period as a bounded duration of observation. Durations are time
differences between pairs of events:
Chronological age is an example of a duration, A = P -- C, for example, and many
other durations may result from taking the time differences between arbitrarily
paired events. For example, adding the event of marriage to the mix, M, results
in several further implied time measures: age at marriage = M -- C, time until
marriage (time married) = M -- P, if P < M (M > P), and so forth.

\section{Constructing temporal identities.}
\label{sec:framework}

We begin this exposition by describing the construction of temporal
identities, largely reproduced in the present section from
\citet{riffe2017unified}. In the following we describe
event-based time frameworks in terms of vector spaces which, via linear transformation, relate the timing of events with durations between events.

\begin{definition} 
  Let $\boldsymbol{p}=(p_1,\ldots,p_n)^\top\in\mathbf{R}^n$ be a vector of $n$
  events or points in time with $n\geq2$. A corresponding vector of durations $\boldsymbol{d}\in\mathbf{R}^m$ is composed by elements of the form $d_{i,j}=p_j-p_i$ for $i=1,\dots,n-1$, $j=2,\dots,n$ and $j>i$.
  \label{def:1}
\end{definition}

The vector of events $\boldsymbol{p}$ can be ordered in an arbitrary way as long as the same elements in $\boldsymbol{p}$ correspond to the same type of event for all observations. A consequence of this is that durations may be either
negative or positive depending on the ordering of events over the life course.

\begin{proposition}
  Given a vector of events $\boldsymbol{p}=(p_1,\ldots,p_n)^\top\in\mathbf{R}^n$, the dimension of the corresponding vector of durations $\boldsymbol{d}\in\mathbf{R}^m$ is $m=n(n-1)/2$.
\end{proposition}

\begin{proof}
  By definition, each element of $\boldsymbol{d}$ is formed by two different elements of $\boldsymbol{p}$. Therefore, the length of $\boldsymbol{d}$ is the number of combinations of 2 different elements from a set of size $n$, such that the order of selection does not matter. From combinatorial theory, it is well known that this value is given by the binomial coefficient $\binom{n}{2}=\frac{n!}{2!(n-2)!}=n(n-1)/2$.
\end{proof}

\begin{proposition}
 For any vector of events $\boldsymbol{p}=(p_1,\ldots,p_n)^\top\in\mathbf{R}^n$, there is always a linear transformation $f:\mathbf{R}^n\to\mathbf{R}^m$ that provides a corresponding vector of durations $\boldsymbol{d}\in\mathbf{R}^m$.
 \label{prop:2}
\end{proposition}

\begin{proof}
 The existence of $f$ is a direct consequence of Definition~\ref{def:1}, given that all the elements of $\boldsymbol{d}$ are a linear combination of elements of $\boldsymbol{p}$. 
\end{proof}

\begin{corollary}
 Given $\boldsymbol{p}=(p_1,\ldots,p_n)^\top\in\mathbf{R}^n$, suppose that $\boldsymbol{d}=(p_2-p_1,\dots,p_n-p_1,p_3-p_2,\dots,p_n-p_2,\dots,p_n-p_{n-1})$. Then, the linear transformation $f:\mathbf{R}^n\to\mathbf{R}^m$ that yields $\boldsymbol{d}$ from $\boldsymbol{p}$ is defined by the $m\times n$ matrix
 %
 \begin{align}
  \boldsymbol{X}_{(m\times n)}&= \left(\begin{array}{cccccc}
			  \multicolumn{1}{c:}{-1}			& \multicolumn{5}{c}{}					\\
			  \multicolumn{1}{c:}{\vdots}	 		& \multicolumn{5}{c}{I_{n-1}}	 			\\  
			  \multicolumn{1}{c:}{-1}			& \multicolumn{5}{c}{}					\\ \hdashline
			  0		& \multicolumn{1}{c:}{-1}	& \multicolumn{4}{c}{}					\\	 
			  \vdots	& \multicolumn{1}{c:}{\vdots}	& \multicolumn{4}{c}{I_{n-2}}  				\\
			  0		& \multicolumn{1}{c:}{-1}	& \multicolumn{4}{c}{}					\\ \hdashline
			  \multicolumn{6}{c}{\cdots}										\\ \hdashline
			  0		& \cdots	& 0		& \multicolumn{1}{c:}{-1}	& 	& 		\\ 
			  0		& \cdots	& 0		& \multicolumn{1}{c:}{-1}	& \multicolumn{2}{c}{\smash{\raisebox{.5\normalbaselineskip}{$I_2$}}}	\\ \hdashline
			  0		& \multicolumn{2}{c}{\cdots}	& 0				& -1	& 1		\\
			\end{array}\right)
  \label{eq:matrix}\ ,
 \end{align}
  %
 such that $\boldsymbol{d}=\boldsymbol{X}\times\boldsymbol{p}$, and where $I_k$ denotes the $k\times k$ identity matrix.
\end{corollary}

These results imply that given an arbitrary set of $n\geq2$ points in time, it
is always possible to calculate the durations between any pair of these points. However, note that matrix $\boldsymbol{X}$ in \eqref{eq:matrix} yields a vector of durations $\boldsymbol{d}\in\mathbf{R}^m$ whose elements are sorted in an arbitrary way. The following statement may be relevant in this regard.

\begin{proposition}
 Given a vector of events $\boldsymbol{p}=(p_1,\ldots,p_n)^\top\in\mathbf{R}^n$, the corresponding vector of durations $\boldsymbol{d}\in\mathbf{R}^m$ is unique, irrespective of the sorting of its elements.
\end{proposition}

\begin{proof}
 Let's suppose that $\boldsymbol{d^1}$ and $\boldsymbol{d^2}$ are two different
 vectors of durations corresponding to the same vector of events
 $\boldsymbol{p}\in\mathbf{R}^n$. Provided that $\boldsymbol{d^1}$ and
 $\boldsymbol{d^2}$ are finite and, by definition, both have dimension $m$ and are formed by the same combinations of elements of $\boldsymbol{p}$, it will always be possible to re-arrange the elements of $\boldsymbol{d^2}$ in the same order as $\boldsymbol{d^1}$ such that $\boldsymbol{d^1}=\boldsymbol{d^2}$.
\end{proof}

This last proposition allows considering $\boldsymbol{X}$ as the matrix defining
the linear transformation between points and durations. Given a vector
$\boldsymbol{p}$ and the corresponding
$\boldsymbol{d}=\boldsymbol{X}\times\boldsymbol{p}$, any differently sorted
vector of durations would be obtained by swapping the rows of $\boldsymbol{X}$.
Further, note that $\boldsymbol{X}$ does not have an inverse matrix, and
therefore there is no linear transformation from durations to events. This is
intuitively straightforward if one thinks that two vectors of events can yield
the same vector of durations. In other words, a particular vector of durations
can come from infinite different vectors of points in time. For instance, using
$\boldsymbol{X}$, the vectors of events $\boldsymbol{p^1}=(1,2,3)$ and
$\boldsymbol{p^2}=(2,3,4)$ both yield $\boldsymbol{d}=(1,2,1)$. 

The relationship between events and durations can be
systematically represented in a series of timelines and graphs that may better
guide intuition.
The joint relationship between events and durations is more explicit and more
compact in a graph representation. As introduced in the following definition, the total number of time measures implied by a set of $n$ events and the corresponding durations is 
$n+m=n+n(n-1)/2=n(n+1)/2$. 

\begin{definition}
Given a vector of events $\boldsymbol{p}=(p_1,\ldots,p_n)^\top\in\mathbf{R}^n$, $n\geq2$, and the corresponding vector of durations $\boldsymbol{d}\in\mathbf{R}^m$, we define the graph of time measures
 $\boldsymbol{G}$ as the graph with $n+m=n+n(n+1)/2$ edges labelled by
 $(\boldsymbol{p},\boldsymbol{d})\in\mathbf{R}^{n(n+1)/2}$ such that the relationships in Definition~\ref{def:1} are preserved.
 \label{def:2}
\end{definition}

Table~3 displays a timeline and a graph for two, three, and four event sets. The central column shows timelines, a familiar linear representation of time, with events
marked with red ticks labelled with $p_1 \ldots p_n$. Durations span
each of the $m$ possible event dyads and are drawn below
the main timeline as curly braces labelled with $d_{1,2} \ldots d_{n-1,n}$. The right
column of Table~3 draws the corresponding graph with a total of $n+1$ vertices and $n+m=n(n+1)/2$ edges
for the elements of both $\boldsymbol{p}$ and $\boldsymbol{d}$. All
events of $\boldsymbol{p}$ connect to a single vertex, and event edges are
indicated in red with red-circled labels. In this rendering, each triangle formed by three
mutually connecting edges represents a triad identity. The top row $n=2$
consists in a single identity. Three and four events imply a total of four and
ten triad identities, respectively, and in general a given higher order identity
will yield $\binom{n+1}{3}$ triad identities.
We call this a temporal plane graph because the triangle resulting from any given triad sub-identity can be extended over
all valid values of its time measures to form a temporal plane, as of the
diagrams in a previous section (not in excerpt). The dimensionality of the extended diagram of a given identity follows from the
number of events from which the identity is derived: $n=2$ produces a
two-dimensional diagram, $n=3$ produces a 3-dimensional diagram, and so forth.

\newcolumntype{S}{ >{\centering\arraybackslash} m{2cm} }
\newcolumntype{D}{ >{\centering\arraybackslash} m{5.4cm} }
\newcolumntype{E}{ >{\centering\arraybackslash} m{3.7cm} }

\begin{table}[ht]
\centering
\caption{Event-duration timeline and graph for two, three, and four event
 sequences.}
\label{tab:timelines}
\makebox[\linewidth][c]{
\begin{tabular}{S D E}
nr. events  & timeline & graph \\
& &\\
   $n = 2 $ & \ttt{1}{1} & \ttt{1}{2} \\
   $n = 3 $ & \ttt{2}{1} & \ttt{2}{2} \\
   $n = 4 $ & \ttt{3}{1} & \ttt{3}{2}
\end{tabular}
}
\end{table}

\begin{definition} 
  We define $P\subseteq\mathbf{R}^n$ as the vector-space (event-space) spanning all possible values vector $\boldsymbol{p}$ may take, and $D\subseteq\mathbf{R}^m$ as the vector space spanning all possible instances of the duration vector $\boldsymbol{d}$. 
  % (FV removed) Let the indices of the bases of $P$ and $D$ correspond to the
  % indices of elements of $\boldsymbol{p}$ and $\boldsymbol{d}$.
  \label{def:1b}
\end{definition}

Just like the APC diagram allows for \emph{all possible} combinations of period,
cohort and age we may consider the vector space $P$ spanning all possible
instances of $\boldsymbol{p}$. The calculation of durations between events as
described in Proposition~\ref{prop:2} can then be understood as a linear
transformation from a vector space $P$ whose bases represent events to a
duration vector space $D$ whose bases represent durations.

\FloatBarrier
\section{Properties of higher order time identities}
Here we list and prove some of the properties of higher-order
identities, such as the number of ways they can be derived, the conditions
for doing so, the number of event and duration measures they contain, and the
number, size, and composition of sub-identities.

\begin{proposition}
\label{mixing} *(Work in progess)* Something that states that you can transform
 $P^n \leftrightarrow M^n$ where the basis vectors of $M$ are a mixture of point dimensions and duration dimensions.
\end{proposition}


\begin{definition}{$\textbf{g}$}
Let $\textbf{g} = \left\{p_{i=1,\ldots,n},
d_{k=1,\ldots,m}\right\}$
\end{definition}

\begin{proposition} There are $b = (n+1)^{(n-1)}$ many
ways to choose $n$ elements out of $\textbf{g}$ whoose linear combination yields the
remaining $m$ elements of $\textbf{g}$.
\end{proposition}

\begin{proof}
This is a case of Cayley's formula \citep{cayley1889}, a result from
graph theory which gives the number of possible trees on $k = n+1$ vertices,
$k^{(k-2)}$. In our case, the fully conected graph $\textbf{K}$ with edges
defined by the elements of $\textbf{g}$ according to the third column of
Table~\ref{tab:timelines}, is complete. By Cayley's fomula, the number of
minimal spanning trees on $\textbf{K}$ is equal to $k^{(k-2)}$. Four different proofs of
this result are given in \citet{aigner2010proofs}. The key is to realize that a
minimal spanning tree (MST) on a complete graph will have $k-1$ edges, connected
to each other and all $k$ vertices. As such, the remaining possible edges are
linear combinations of any given MST.
\end{proof}

\begin{corollary}
Each set of $n$ elements from $\textbf{g}$, $\textbf{b}'$ whose linear
transformation yields the remaining $m$ elements of $\textbf{g}$ includes at
least one element of $\textbf{p}$.
\end{corollary}

\begin{proof}
One of the vertices of $\textbf{K}$, say the $k^{\text{th}}$ vertex, is
connected only to edges labelled by the elements of $\textbf{p}$. Since a
spanning tree of $\textbf{K}$ must connect to this vertex to fully connect
$\textbf{K}$, all $b$ valid spanning trees must contain at least one edge
labelled by an element of $\textbf{p}$.
\end{proof}

\begin{definition}{$\textbf{G}$}
Let's define $\textbf{G}$ as the identity implied by $\textbf{g}$ whose
graph is $\textbf{K}$.
\end{definition}

\begin{proposition}
An identity $\textbf{G}$ implies a total of $\binom{n+1}{3}$ triad
sub-identities.
\end{proposition}

\begin{proof}
Any set of three vertices from $\textbf{K}$ forms a complete subgraph, and any
complete subgraph implies an identity between its labelled edges. $\textbf{K}$
has $n+1$ vertices, and therefore there are $\binom{n+1}{3}$ ways to select
three vertices from $\textbf{K}$, hence $\textbf{G}$ implies the same number of
triad subidentities.
\end{proof}

\begin{corollary}
Of the $\binom{n+1}{3}$ triad identies implied by $\textbf{G}$, $\binom{n}{2}$
contain exactly two events and one duration.
\end{corollary}

\begin{proof}
This follows by noting that the $n$ event-labelled edges in $\textbf{K}$ connect
to a single vertex. Selecting any two of these $n$ event-labelled edges implies
a tree on three vertices, whose full connection implies a triad identity
composed of the two event edges and one duration edge defined as the time-difference of the
former two. There are $\binom{n}{2}$ ways to select two of of the $n$ event
edges.
\end{proof}

\begin{corollary}
For $n \ge 3$, of the $\binom{n+1}{3}$ triad identies implied by $\textbf{G}$,
$\binom{n+1}{3} - \binom{n}{2} = \binom{n}{3}$ are composed of exactly three
durations.
\end{corollary}

\begin{proof}
This is equivalent to deleting the vertex $k^{\text{th}}$ from $\textbf{K}$, the
vertex that connects only to event-labelled edges, which is constructed
following the middle column of graphs from Table~\ref{tab:timelines} with
vertex labels ignored. This graph has $n$ total vertices, and any set of 3
vertices implies an identity between its three labelled edges, which in this
case by definition can only consist of durations.
\end{proof}

For example, in the demographic time identity there are $n=3$ event measures.
Thus of the $\binom{4}{3} = 4$ triad identities implied $\binom{3}{3} = 1$ of
these identities consist in durations only (TAL). Notice that the measures T and
A change over the lifecourse of an individual, whereas their sum L is fixed.

\begin{definition}{$\textbf{d}_t$} For $\textbf{p}$ that include period
itself, let $\textbf{d}_t$ be the set of duration time measures that change over
the life course and $\textbf{d}_f$ consist in those durations that are fixed attributed of an individual. By
definition, $\textbf{d}_t \cup \textbf{d}_f = \textbf{d}$.
\end{definition}

\begin{corollary}
For $n \ge 4$ and For $\textbf{p}$ that include period
itself, of the $d'=\binom{n}{3}$ triad identies whose edges are
labelled only by the elements of $\textbf{d}$, $d^t=\binom{d'}{2}$ of these identities
consist in exactly two elements of $\textbf{d}_t$ and one element of
$\textbf{d}_f$, while $d'-d^t$ of the duration-only triad identities consist in
relationships between three elements of $\textbf{d}_f$.
\end{corollary}

\begin{proof}
$n-2$ of the edges in $K$ are labelled with the elements of $\textbf{d}_t$, and
these all connect to the same vertex. There are therefore $\binom{n-2}{2}$ ways
to form triad identities with them. The third element of each of these
identities cannot connect to the same vertex, and so must be a member of
$\textbf{d}_f$. 
\end{proof}

\begin{proposition}
In general, the number of subidentities of size $h$ in $\textbf{G}$ is equal to
$\binom{n+1}{n+1-h} \quad \forall \, h \le n$.
\end{proposition}

\begin{proof}
Vertex deletion on a complete graph results in a complete subgraph. Therefore,
the number of possible complete subgraphs with $h$ vertices is a function of the
number of ways that $n+1-h$ vertices can be deleted from $\textbf{K}$, which is
$\binom{n+1}{n+1-h}$. The labelled edges of each possible complete subgraph
defined in this way represent subidentities.
\end{proof}

For example, from the tetrahedral graph in Fig.~\ref{fig:tet}, we may
delete the vertex that joins the edges labelled A, T, and P, which in effect
deletes these edges, leaving us with the CDL identity. 

\begin{corollary}
Each time measure in $\textbf{G}$ is a member of $n-1$ triad subidentities.
\end{corollary}

\begin{proof}
In the graph $\textbf{K}$, an edge labelled by a given time measure connects to
two of the $n+1$ vertices. A full connection to any other vertex yields a
complete subgraph with three edges, representing a triad identity. There are
$n-1$ remainign vertices that can be connected to, ergo the given edge is a
member of $n-1$ triad subidentities.
\end{proof}


\begin{backmatter}

%\section*{Competing interests}
%  The authors declare that they have no competing interests.

%\section*{Author's contributions}
%    Text for this section \ldots

\section*{Acknowledgements}
We wish to thank Francisco Villavicencio for comments that improved our
exposition.

\bibliographystyle{chicago}
\bibliography{references}  
\end{backmatter}
\end{document}
